% PLANTILLA APA7
% Creado por: Andres Bergsneider
% GNU General Public License
% Última actualización: 23/07/2021

% Preámbulo
\documentclass[doc, 12pt, letterpaper, donotrepeattitle, floatsintext, natbib]{apa7}
\usepackage[utf8]{inputenc}
\usepackage{comment}
\usepackage{marvosym}
\usepackage{graphicx}
\usepackage{float}
\usepackage[normalem]{ulem}
\usepackage[spanish]{babel}
\usepackage{setspace}
\selectlanguage{spanish}
\useunder{\uline}{\ul}{}
\newcommand{\myparagraph}[1]{\paragraph{#1}\mbox{}\\}

% Portada
%\thispagestyle{empty}
\title{\Large Reflexión sobre la actual situación en Colombia}
\shorttitle{Reflexión sobre la actual situación en Colombia}
\author{Natalia Rodríguez Bergsneider} 
\affiliation{Universidad de los Andes\\ Constitución y Democracia DERE 1300\\ Codigo: 201923806}
\course{Código: 201923806}
%\professor{Nombre del docente}
%\duedate{Fecha}

% De aqui para abajo empieza el documento [AB]
\begin{document}
\vspace*{-2cm}      % removes vertical space on title section [AB]
\maketitle          % PORTADA, quitar "%" al principio para activar


    % Índices
%\pagenumbering{roman}
    % Contenido
%\renewcommand\contentsname{\largeÍndice}
%\tableofcontents
%\setcounter{tocdepth}{2}
%\newpage
    % Fíguras
%\renewcommand{\listfigurename}{\largeÍndice de fíguras}
%\listoffigures
%\newpage
    % Tablas
%\renewcommand{\listtablename}{\largeÍndice de tablas}
%\listoftables
%\newpage
%\noindent Universidad de los Andes\\
%\noindent Constitución y Democracia DERE 1300\\
%\noindent Natalia Rodríguez Bergsneider\\
%\noindent Código: 201923806\\

% Cuerpo
\pagenumbering{arabic}

%\section{\large Reflexión sobre la actual situación en Colombia}
\doublespacing % delete for single spacing [AB]
Para comenzar, es importante entender que la situación que está atravesando Colombia es quizá una de las situaciones más difíciles y complicadas que le ha tocado vivir al país en mucho tiempo. Difícil y complicada por la innumerable cantidad de factores que están metidos en este problema: aspectos económicos, éticos, legales, políticos, étnicos, y un gran etc. El objetivo de este ensayo es hacer una reflexión sobre la situación en Colombia y para tratar de responder a la pregunta... ¿qué puede hacerse a futuro para mejorar esta situación? Primero, hablaré un poco sobre la situación actual con el fin de contextualizar al leyente. Segundo, mencionaré algunos aspectos de las lecturas y temas relevantes vistos a lo largo del semestre y finalmente, me apoyaré de ciertas campañas y datos obtenidos de diversas fuentes para sustentar mis argumentos. Lo anterior, con el fin de dar una posible solución a esta situación.
\noindent \maskCitet{cervantes1999}\\


\subsection{Paro Nacional} 
En primer lugar, el pasado 28 de abril de 2021 miles de personas en Colombia y a nivel mundial salieron a las calles en forma de protesta y rechazo al proyecto de reforma tributaria del gobierno del actual presidente Iván Duque. Si bien todo empezó de manera pacífica al finalizar el día y con la primera noticia donde confirmaron la muerte de una persona todo se torno oscuro. En diferentes ciudades de Colombia como \ Bogotá, Medellín y Cali hubieron diferentes disturbios, incendios, actos de vandalismo y hasta heridos. Por lo tanto, ¿qué es lo que está pasando?...


\subsection{Aspectos económicos} 
En primera instancia, el gobierno que preside el presidente Duque, justamente en el momento en que el país y el mundo entero, está pasando por una pandemia espantosa debido al COVID-19 que tenía y tiene nuestra economía por los suelos, decide aplicar una reforma fiscal que va directamente contra las posibilidades económicas no solamente de la clase media, sino de toda la nación. Escogió, por sugerencia de su entonces ministro de Hacienda, el señor Alberto Carrasquilla, el momento más inoportuno, tal vez porque hay un impresionante divorcio entre la clase dirigente y los requerimientos del ciudadano de a pie: la llamada clase media, y los estratos más bajos de la población.

Cabe mencionar además que aunque se esté o no a favor de la Reforma Tributaria, lo importante acá es lo inoportuno del momento. Personalmente, creo que es necesaria una reforma porque, siendo Colombia un país del tercer mundo, que no tiene una producción comercial e industrial plenamente satisfactoria, sobrevivimos en buena parte mediante la venta de nuestros recursos naturales: petróleo, carbón, oro, níquel etc. Que le ayude a salir de su precariedad económica, con frecuencia tiene que sobrevivir de préstamos internacionales. A manera de información y con motivo de la caída de los precios del petróleo, los últimos datos según la revista La República muestran que, hasta el año 2020 la deuda externa acumulada de Colombia llegaba a los 142.800 millones de dólares (equivalente al 54.8
\% de nuestro Producto interno bruto (PIB). Y que si o si hay que pagar esa deuda, sino a Colombia no se le volverá a prestar dinero.

\subsection{Aspectos éticos}
Si a lo anterior se le suma la situación de la preocupante corrupción que impera en el país, que según la revista Portafolio se habla de 50 billones de pesos por año, esto está creado el caldo de cultivo perfecto para un estallido social con consecuencias muy peligrosas. La nación, en términos generales, ha venido durante años soportando escándalo tras escándalo, donde es de conocimiento público la “evaporación” de miles y miles de millones de pesos gracias al desgreño administrativo. Pero eso sí, para beneficio de unos pocos y en donde brilla con luz propia la corrupción. Además, con el estallido de la fracasada Reforma Tributaria, todo el rencor, la ira y la desesperación estalló.

Como consecuencia del anterior estado de cosas expuesto en el párrafo anterior, vino el paro, que en sus orígenes no tenía más impulso que la comentada ira, rencor y desesperación, pero que, con el paso de los días, encontró apoyo y guía en políticos de la corriente opositora del actual gobierno, paralelamente con dirigentes gremiales que piensan aprovecharse de estas circunstancias, para obtener dividendos económicos significativos.

\subsection{Aspectos legales}
Asimismo, es importante mencionar la grave situación de violación de Derechos Humanos en Colombia durante las jornadas de protestas. Esto debido a que tanto la policía como el Esmad han agredido de manera brutal a las personas que han participado en ellas y como ejemplo de la brutalidad y el exceso de fuerza y arbitrariedad por parte de la fuerza pública, la campaña Defender la libertad: ha registrado que hasta el día 3 de mayo del presente año un total de:
%\begin{textit}
\begin{itemize}
    \item \textbf{18 personas han sido asesinadas} presuntamente por el accionar de la Policía.
    \item \textbf{988 personas han sido detenidas}, gran parte de ellas por medio de procedimientos arbitrarios, siendo sometidas a tortura y/o tratos crueles.
    \item \textbf{8 allanamientos} que fueron declarados ilegales, incluyendo las capturas asociadas.
    \item \textbf{Se han presentado 398 denuncias} por abusos de poder, autoridad, agresiones y violencia policial.
    \item \textbf{11 personas fueron víctimas de violencias} basadas en género ejercidas por la misma institución.
\end{itemize}

%\end{textit}

Por otro lado, con el ánimo de salir del atolladero en que se encuentra el país, el presidente Duque ha llamado a conversaciones con los diferentes sectores que están involucrados en el paro, conversaciones que como todo conflicto que se desea solucionar, tiene una clara condición: el diálogo y la búsqueda de soluciones. Al diálogo, obviamente, se opone la imposición radical sin concesiones. A tal punto que, si quienes asisten a estos diálogos, no los aceptan, pretendiendo imponer, a como dé lugar, la totalidad de sus puntos de vista, aun en contraposición a las leyes y a la Constitución, el fracaso es concluyente. Con consecuencias como las que ya se observaron: suspensión del diálogo y retiro de los asistentes.

\subsection{Aspectos políticos}
Este aspecto, cuyo soporte, en parte, tiene que ver con lo ya expuesto en los Aspectos Éticos, hace que nos encontremos frente a un panorama desolador en donde diferentes corrientes políticas y económicas están enredadas buscando una esquiva solución, con el agravante de que no se tiene un norte definido. Por un lado, Duque, en su calidad de Presidente del país, incluyendo la clase dirigente donde desafortunadamente existen sujetos corruptos usufructuarios del mencionado desgreño administrativo y, por el otro lado, un grupo de parlamentarios de izquierda, comandados por Gustavo Petro (el contendor del actual Presidente en la pasada campaña presidencial). Y en ese mundo revuelto (para mirar con mucha preocupación), ahora tenemos los aspectos políticos en donde la pesca en río revuelto está a la orden del día: las derechas. que siempre han gobernado este país y las izquierdas que, a su vez, han estado en la oposición. Resumiendo: La derecha comandada por Duque, como cabeza visible, Versus la oposición de izquierda, dirigida por Gustavo Petro. Por último es importante no olvidar el problema de las etnias (mingas indígenas), que participan en los bloqueos, agravando el incierto panorama económico y social del país.

En conclusión, y con lo analizado previamente, podemos concluir que son muchos los factores que influyen en que hoy en día Colombia sea un país con tantos problemas. Factores que acrecientan la desigualdad, el odio, la ira, el rencor, el miedo, la pobreza, la inseguridad y la insatisfacción. Si bien es cierto, para nadie es un secreto que de proseguir así, no solamente es imposible que un país progrese, sino que los desequilibrios sociales generados por estas situaciones, pueden inducir a problemas superiores. Por lo tanto, se puede decir con certeza que Colombia debe llegar a la raíz de todos estos problemas y empezar hacer cambios que beneficien a toda la población. Cambios como: la eliminación de la violencia, mayores oportunidades laborales, justicia, cambios en el gobierno, menos desigualdad, la erradicación total de la corrupción ya que en un país como Colombia, en donde esto afecta en todos los niveles no habrá ningún progreso sin ningún cambio y se verá siempre como un intento fallido. Es por esta razón que en este ensayo más allá de apresar los causantes y señalar a un culpable, trae consigo la finalidad de introducir un debate acerca de cómo los colombianos podemos aportar. ¿Y usted qué puede hacer de manera individual para reconstruir a nuestro país?


\newpage
% Referencias
\renewcommand\refname{\large\textbf{Referencias bibliográficas:}}
\bibliography{mibibliografia}

\end{document}